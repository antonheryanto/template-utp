%Chapter 3
%-----------------------------------------------
%-----------------------------------------------
%-----------------------------------------------
%\pagestyle{plain}\cleardoublepage %To Add empty page and start on odd number (SOFTBOUND)%%
\chapter{Methodology}
\label{ch3}
\section{Introduction}
As a prior work, the delay behavior of paths affected by ROFs  with respect to $V_{DD}$ has been investigated. The experimental results showed that up to 3 resistance ranges per fault location and test pattern can exhibit different faulty behaviors. Additionally, the ROF detectability with respect to $V_{DD}$ in multiple $V_{DD}$ environments for different fault locations and test patterns was investigated. The experimental results showed that the detectability with respect to $V_{DD}$ varies across test patterns, designs and technologies. This means that each ROF could have a different detectable resistance range coverage with $V_{DD}$ and this implies a difference in the resistance detection threshold at each $V_{DD}$. \par
In this section, the experimental setup and results will be discussed, followed by an experiment on an inverter chain design.
\section{Prior Work}
\label{sec_ResIntnResDetThr}

As a prior work, the delay behavior of paths affected by ROFs  with respect to $V_{DD}$ has been investigated. The experimental results showed that up to 3 resistance ranges per fault location and test pattern can exhibit different faulty behaviors. Additionally, the ROF detectability with respect to $V_{DD}$ in multiple $V_{DD}$ environments for different fault locations and test patterns was investigated. The experimental results showed that the detectability with respect to $V_{DD}$ varies across test patterns, designs and technologies. This means that each ROF could have a different detectable resistance range coverage with $V_{DD}$ and this implies a difference in the resistance detection threshold at each $V_{DD}$. \par
In this section, the experimental setup and results will be discussed, followed by an experiment on an inverter chain design.

\subsection{Experiments}
\label{JP2_subsec_BenchmarkExp}


Experiments on benchmark circuits~\cite{F.Brglez-85,F.Brglez-89,F.corno-00} were performed to determine the identifiable number of behaviors per fault for each circuit. The list of benchmark circuits used is shown in Table~\ref{tb:iscas85}Additionally, the average detectable resistance range per $V_{DD}$ for a large number of fault locations was computed for each circuit. Spice-level simulations were used. The delays of faulty paths were calculated at the different $V_{DD}$ values. If two ROFs at the same $V_{DD}$  had the largest delay value, then they were grouped. 


\begin{table}[hbtp]
\centering
\caption{Benchmark circuit characteristics used in the experiments}
\label{tb:iscas85}
\begin{tabular}{|c||c|c|c|c|} \hline
Circuit & Circuit & Total & Input & Output  \\
Name    & Function & Gates & Lines & Lines \\ \hline
        &          &       &      &         \\
C17	& N/A & 5 & 5   & 7       \\
        &          &       &      &         \\
S27	& N/A & 10 (3 flip-flops) & 4   & 1       \\
        &          &       &      &         \\
B01	& Finite State Machine & 49 (5 flip-flops) & 2   & 2       \\
        &          &       &      &         \\
C432	& Priority Decoder & 160 (18 EXOR)  & 36   & 7       \\
        &          &       &      &         \\
C499$^1$ & ECAT     & 202 (104 EXOR) & 41   & 32     \\
        &          &       &      &         \\
C880    & ALU and Control  & 383   & 60   & 26      \\
        &          &       &      &         \\
C1355$^1$ & ECAT     & 546   & 41   & 32      \\
        &          &       &      &         \\
C1908   & ECAT     & 880   & 33   & 25      \\
        &          &       &      &        \\
\hline
\end{tabular}

\raggedright 
\hspace{0.8in}
\parbox{4.8in}{  
$^1$Circuits C499 and C1355 are functionally equivalent.
All EXOR gates of C499 have been expanded into their
4-NAND gate equivalents in C1355. \cite{M.Pedram-02}
}

\end{table}
%-----------------------------------------------%-----------------------------------------------

